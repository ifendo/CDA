% Options for packages loaded elsewhere
\PassOptionsToPackage{unicode}{hyperref}
\PassOptionsToPackage{hyphens}{url}
%
\documentclass[
]{book}
\usepackage{lmodern}
\usepackage{amssymb,amsmath}
\usepackage{ifxetex,ifluatex}
\ifnum 0\ifxetex 1\fi\ifluatex 1\fi=0 % if pdftex
  \usepackage[T1]{fontenc}
  \usepackage[utf8]{inputenc}
  \usepackage{textcomp} % provide euro and other symbols
\else % if luatex or xetex
  \usepackage{unicode-math}
  \defaultfontfeatures{Scale=MatchLowercase}
  \defaultfontfeatures[\rmfamily]{Ligatures=TeX,Scale=1}
\fi
% Use upquote if available, for straight quotes in verbatim environments
\IfFileExists{upquote.sty}{\usepackage{upquote}}{}
\IfFileExists{microtype.sty}{% use microtype if available
  \usepackage[]{microtype}
  \UseMicrotypeSet[protrusion]{basicmath} % disable protrusion for tt fonts
}{}
\makeatletter
\@ifundefined{KOMAClassName}{% if non-KOMA class
  \IfFileExists{parskip.sty}{%
    \usepackage{parskip}
  }{% else
    \setlength{\parindent}{0pt}
    \setlength{\parskip}{6pt plus 2pt minus 1pt}}
}{% if KOMA class
  \KOMAoptions{parskip=half}}
\makeatother
\usepackage{xcolor}
\IfFileExists{xurl.sty}{\usepackage{xurl}}{} % add URL line breaks if available
\IfFileExists{bookmark.sty}{\usepackage{bookmark}}{\usepackage{hyperref}}
\hypersetup{
  pdftitle={Clinical Data Analysis},
  pdfauthor={Jie Wang},
  hidelinks,
  pdfcreator={LaTeX via pandoc}}
\urlstyle{same} % disable monospaced font for URLs
\usepackage{color}
\usepackage{fancyvrb}
\newcommand{\VerbBar}{|}
\newcommand{\VERB}{\Verb[commandchars=\\\{\}]}
\DefineVerbatimEnvironment{Highlighting}{Verbatim}{commandchars=\\\{\}}
% Add ',fontsize=\small' for more characters per line
\usepackage{framed}
\definecolor{shadecolor}{RGB}{248,248,248}
\newenvironment{Shaded}{\begin{snugshade}}{\end{snugshade}}
\newcommand{\AlertTok}[1]{\textcolor[rgb]{0.94,0.16,0.16}{#1}}
\newcommand{\AnnotationTok}[1]{\textcolor[rgb]{0.56,0.35,0.01}{\textbf{\textit{#1}}}}
\newcommand{\AttributeTok}[1]{\textcolor[rgb]{0.77,0.63,0.00}{#1}}
\newcommand{\BaseNTok}[1]{\textcolor[rgb]{0.00,0.00,0.81}{#1}}
\newcommand{\BuiltInTok}[1]{#1}
\newcommand{\CharTok}[1]{\textcolor[rgb]{0.31,0.60,0.02}{#1}}
\newcommand{\CommentTok}[1]{\textcolor[rgb]{0.56,0.35,0.01}{\textit{#1}}}
\newcommand{\CommentVarTok}[1]{\textcolor[rgb]{0.56,0.35,0.01}{\textbf{\textit{#1}}}}
\newcommand{\ConstantTok}[1]{\textcolor[rgb]{0.00,0.00,0.00}{#1}}
\newcommand{\ControlFlowTok}[1]{\textcolor[rgb]{0.13,0.29,0.53}{\textbf{#1}}}
\newcommand{\DataTypeTok}[1]{\textcolor[rgb]{0.13,0.29,0.53}{#1}}
\newcommand{\DecValTok}[1]{\textcolor[rgb]{0.00,0.00,0.81}{#1}}
\newcommand{\DocumentationTok}[1]{\textcolor[rgb]{0.56,0.35,0.01}{\textbf{\textit{#1}}}}
\newcommand{\ErrorTok}[1]{\textcolor[rgb]{0.64,0.00,0.00}{\textbf{#1}}}
\newcommand{\ExtensionTok}[1]{#1}
\newcommand{\FloatTok}[1]{\textcolor[rgb]{0.00,0.00,0.81}{#1}}
\newcommand{\FunctionTok}[1]{\textcolor[rgb]{0.00,0.00,0.00}{#1}}
\newcommand{\ImportTok}[1]{#1}
\newcommand{\InformationTok}[1]{\textcolor[rgb]{0.56,0.35,0.01}{\textbf{\textit{#1}}}}
\newcommand{\KeywordTok}[1]{\textcolor[rgb]{0.13,0.29,0.53}{\textbf{#1}}}
\newcommand{\NormalTok}[1]{#1}
\newcommand{\OperatorTok}[1]{\textcolor[rgb]{0.81,0.36,0.00}{\textbf{#1}}}
\newcommand{\OtherTok}[1]{\textcolor[rgb]{0.56,0.35,0.01}{#1}}
\newcommand{\PreprocessorTok}[1]{\textcolor[rgb]{0.56,0.35,0.01}{\textit{#1}}}
\newcommand{\RegionMarkerTok}[1]{#1}
\newcommand{\SpecialCharTok}[1]{\textcolor[rgb]{0.00,0.00,0.00}{#1}}
\newcommand{\SpecialStringTok}[1]{\textcolor[rgb]{0.31,0.60,0.02}{#1}}
\newcommand{\StringTok}[1]{\textcolor[rgb]{0.31,0.60,0.02}{#1}}
\newcommand{\VariableTok}[1]{\textcolor[rgb]{0.00,0.00,0.00}{#1}}
\newcommand{\VerbatimStringTok}[1]{\textcolor[rgb]{0.31,0.60,0.02}{#1}}
\newcommand{\WarningTok}[1]{\textcolor[rgb]{0.56,0.35,0.01}{\textbf{\textit{#1}}}}
\usepackage{longtable,booktabs}
% Correct order of tables after \paragraph or \subparagraph
\usepackage{etoolbox}
\makeatletter
\patchcmd\longtable{\par}{\if@noskipsec\mbox{}\fi\par}{}{}
\makeatother
% Allow footnotes in longtable head/foot
\IfFileExists{footnotehyper.sty}{\usepackage{footnotehyper}}{\usepackage{footnote}}
\makesavenoteenv{longtable}
\usepackage{graphicx,grffile}
\makeatletter
\def\maxwidth{\ifdim\Gin@nat@width>\linewidth\linewidth\else\Gin@nat@width\fi}
\def\maxheight{\ifdim\Gin@nat@height>\textheight\textheight\else\Gin@nat@height\fi}
\makeatother
% Scale images if necessary, so that they will not overflow the page
% margins by default, and it is still possible to overwrite the defaults
% using explicit options in \includegraphics[width, height, ...]{}
\setkeys{Gin}{width=\maxwidth,height=\maxheight,keepaspectratio}
% Set default figure placement to htbp
\makeatletter
\def\fps@figure{htbp}
\makeatother
\setlength{\emergencystretch}{3em} % prevent overfull lines
\providecommand{\tightlist}{%
  \setlength{\itemsep}{0pt}\setlength{\parskip}{0pt}}
\setcounter{secnumdepth}{5}
\usepackage{booktabs}
\usepackage[]{natbib}
\bibliographystyle{apalike}

\title{Clinical Data Analysis}
\author{Jie Wang}
\date{2020-12-13}

\begin{document}
\maketitle

{
\setcounter{tocdepth}{1}
\tableofcontents
}
\hypertarget{prerequisites}{%
\chapter{Prerequisites}\label{prerequisites}}

This is a \emph{sample} book written in \textbf{Markdown}. You can use anything that Pandoc's Markdown supports, e.g., a math equation \(a^2 + b^2 = c^2\).

The \textbf{bookdown} package can be installed from CRAN or Github:

\begin{Shaded}
\begin{Highlighting}[]
\KeywordTok{install.packages}\NormalTok{(}\StringTok{"bookdown"}\NormalTok{)}
\CommentTok{# or the development version}
\CommentTok{# devtools::install_github("rstudio/bookdown")}
\end{Highlighting}
\end{Shaded}

Remember each Rmd file contains one and only one chapter, and a chapter is defined by the first-level heading \texttt{\#}.

To compile this example to PDF, you need XeLaTeX. You are recommended to install TinyTeX (which includes XeLaTeX): \url{https://yihui.org/tinytex/}.

\hypertarget{intro}{%
\chapter{Introduction}\label{intro}}

You can label chapter and section titles using \texttt{\{\#label\}} after them, e.g., we can reference Chapter \ref{intro}. If you do not manually label them, there will be automatic labels anyway, e.g., Chapter \ref{methods}.

Figures and tables with captions will be placed in \texttt{figure} and \texttt{table} environments, respectively.

\begin{Shaded}
\begin{Highlighting}[]
\KeywordTok{par}\NormalTok{(}\DataTypeTok{mar =} \KeywordTok{c}\NormalTok{(}\DecValTok{4}\NormalTok{, }\DecValTok{4}\NormalTok{, }\FloatTok{.1}\NormalTok{, }\FloatTok{.1}\NormalTok{))}
\KeywordTok{plot}\NormalTok{(pressure, }\DataTypeTok{type =} \StringTok{'b'}\NormalTok{, }\DataTypeTok{pch =} \DecValTok{19}\NormalTok{)}
\end{Highlighting}
\end{Shaded}

\begin{figure}

{\centering \includegraphics[width=0.8\linewidth]{CDA_files/figure-latex/nice-fig-1} 

}

\caption{Here is a nice figure!}\label{fig:nice-fig}
\end{figure}

Reference a figure by its code chunk label with the \texttt{fig:} prefix, e.g., see Figure \ref{fig:nice-fig}. Similarly, you can reference tables generated from \texttt{knitr::kable()}, e.g., see Table \ref{tab:nice-tab}.

\begin{Shaded}
\begin{Highlighting}[]
\NormalTok{knitr}\OperatorTok{::}\KeywordTok{kable}\NormalTok{(}
  \KeywordTok{head}\NormalTok{(iris, }\DecValTok{20}\NormalTok{), }\DataTypeTok{caption =} \StringTok{'Here is a nice table!'}\NormalTok{,}
  \DataTypeTok{booktabs =} \OtherTok{TRUE}
\NormalTok{)}
\end{Highlighting}
\end{Shaded}

\begin{table}

\caption{\label{tab:nice-tab}Here is a nice table!}
\centering
\begin{tabular}[t]{rrrrl}
\toprule
Sepal.Length & Sepal.Width & Petal.Length & Petal.Width & Species\\
\midrule
5.1 & 3.5 & 1.4 & 0.2 & setosa\\
4.9 & 3.0 & 1.4 & 0.2 & setosa\\
4.7 & 3.2 & 1.3 & 0.2 & setosa\\
4.6 & 3.1 & 1.5 & 0.2 & setosa\\
5.0 & 3.6 & 1.4 & 0.2 & setosa\\
\addlinespace
5.4 & 3.9 & 1.7 & 0.4 & setosa\\
4.6 & 3.4 & 1.4 & 0.3 & setosa\\
5.0 & 3.4 & 1.5 & 0.2 & setosa\\
4.4 & 2.9 & 1.4 & 0.2 & setosa\\
4.9 & 3.1 & 1.5 & 0.1 & setosa\\
\addlinespace
5.4 & 3.7 & 1.5 & 0.2 & setosa\\
4.8 & 3.4 & 1.6 & 0.2 & setosa\\
4.8 & 3.0 & 1.4 & 0.1 & setosa\\
4.3 & 3.0 & 1.1 & 0.1 & setosa\\
5.8 & 4.0 & 1.2 & 0.2 & setosa\\
\addlinespace
5.7 & 4.4 & 1.5 & 0.4 & setosa\\
5.4 & 3.9 & 1.3 & 0.4 & setosa\\
5.1 & 3.5 & 1.4 & 0.3 & setosa\\
5.7 & 3.8 & 1.7 & 0.3 & setosa\\
5.1 & 3.8 & 1.5 & 0.3 & setosa\\
\bottomrule
\end{tabular}
\end{table}

You can write citations, too. For example, we are using the \textbf{bookdown} package \citep{R-bookdown} in this sample book, which was built on top of R Markdown and \textbf{knitr} \citep{xie2015}.

\hypertarget{literature}{%
\chapter{Literature}\label{literature}}

Here is a review of existing methods.

\hypertarget{methods}{%
\chapter{Methods}\label{methods}}

We describe our methods in this chapter.

\hypertarget{applications}{%
\chapter{Applications}\label{applications}}

Some \emph{significant} applications are demonstrated in this chapter.

\hypertarget{example-one}{%
\section{Example one}\label{example-one}}

\hypertarget{example-two}{%
\section{Example two}\label{example-two}}

\hypertarget{final-words}{%
\chapter{Final Words}\label{final-words}}

We have finished a nice book.

\hypertarget{the-binomial-test}{%
\chapter{The Binomial Test}\label{the-binomial-test}}

\hypertarget{overview}{%
\section{Overview}\label{overview}}

The \textbf{binomial test} is used to make inferences about a proportion or response
rate based on a series of independent observations, each resulting in one of two
possible mutually exclusive outcomes, such as:

\begin{verbatim}
+ response to treatment vs. no response

+ cure or no cure

+ survival or death

+ event vs non-event (in general)
\end{verbatim}

The total number of \emph{events} in n observations, X, follows the binomial probability
distribution. Intuitively, the sample proportion, X/n, would be a good estimate of
the unknown population proportion, p.~Statistically, it is the best estimate.

You want to determine whether the population proportion, p, differs from a
hypothesized value, p\textsubscript{0}. If the unknown proportion, p, equals p\textsubscript{0}, then the estimated proportion, X/n,
should be close to p\textsubscript{0}, i.e., X should be close to n * p\textsubscript{0}. When p differs from p\textsubscript{0},
X might be much larger or smaller than n * p\textsubscript{0}.

SAS function, \textbf{probbnml()} can be used to determine X\textsubscript{L} and X\textsubscript{U} (lower limit and
upper limit)

\hypertarget{proc-freq-to-calculate-wald-ci}{%
\chapter{\#\# proc freq to calculate Wald CI}\label{proc-freq-to-calculate-wald-ci}}

\hypertarget{normal-approximation}{%
\section{Normal Approximation}\label{normal-approximation}}

For larger values of n and non-extreme values of p, a binomial response, X, can be
approximated by a normal distribution with mean n * p and variance n * p * (1-p). This
approximation improves as n gets larger or as p gets closer to 0.5

\hypertarget{a-proc-freq-example}{%
\section{A proc freq example}\label{a-proc-freq-example}}

\begin{Shaded}
\begin{Highlighting}[]
\NormalTok{data acr20;}
\BaseNTok{    input patient $ avalc $ @@;}
\BaseNTok{    cards;}
\BaseNTok{    1 Yes 2 No}
\BaseNTok{    3 Yes 4 No}
\BaseNTok{    5 Yes 6 Yes}
\BaseNTok{    7 No  8 Yes}
\BaseNTok{    9 No  10 No}
\BaseNTok{    11 Yes 12 No}
\BaseNTok{    13 Yes 14 No}
\BaseNTok{    15 Yes 16 No}
\BaseNTok{    17 No  18 Yes}
\BaseNTok{    19 Yes 20 No}
\BaseNTok{    21 Yes 22 Yes}
\BaseNTok{    23 No 24 Yes}
\BaseNTok{    25 Yes}
\BaseNTok{    ;}
\NormalTok{run;}

\NormalTok{data acr20a;}
\BaseNTok{    set acr20;}
\BaseNTok{    avalc=ifc(avalc="Yes", "1Yes", "2No");}
\NormalTok{run;}

\NormalTok{proc freq data=acr20a;}
\BaseNTok{    tables avalc / binomialc (p = 0.4) alpha=0.05;}
\BaseNTok{    exact binomial;}
\BaseNTok{    title1 "Binomial Test";}
\NormalTok{run;}
\end{Highlighting}
\end{Shaded}

\hypertarget{a-real-example-from-trial}{%
\subsection{A real example from trial}\label{a-real-example-from-trial}}

\includegraphics[width=13.01in]{images/wald_ci}

\hypertarget{proc-freq-to-calculate-wald-ci-1}{%
\subsubsection{proc freq to calculate Wald CI}\label{proc-freq-to-calculate-wald-ci-1}}

\begin{Shaded}
\begin{Highlighting}[]
\NormalTok{data resp;}
\BaseNTok{    input avisitn avisit $ trt01pn avalc $ count;}
\BaseNTok{    cards;}
\BaseNTok{    20052 Week_52 1 N 124}
\BaseNTok{    20052 Week_52 1 Y 112}
\BaseNTok{    20052 Week_52 2 N 97}
\BaseNTok{    20052 Week_52 2 Y 131}
\BaseNTok{    20052 Week_52 3 N 94}
\BaseNTok{    20052 Week_52 3 Y 134}
\BaseNTok{    20068 Week_68 1 N 113}
\BaseNTok{    20068 Week_68 1 Y 123}
\BaseNTok{    20068 Week_68 2 N 91}
\BaseNTok{    20068 Week_68 2 Y 137}
\BaseNTok{    20068 Week_68 3 N 85}
\BaseNTok{    20068 Week_68 3 Y 143}
\BaseNTok{    ;}
\NormalTok{run;}

\NormalTok{proc sort data=resp;}
\BaseNTok{    by avisitn avisit trt01pn;}
\NormalTok{run;}

\NormalTok{ods output BinomialCLs=bincl;  }
\NormalTok{proc freq data=resp;}
\BaseNTok{    by avisitn avisit trt01pn;}
\BaseNTok{    table avalc/binomial(level = "Y" CL=WALD(CORRECT)); }
\BaseNTok{    weight count;}
\NormalTok{run;}

\NormalTok{data resp2;}
\BaseNTok{    set bincl;}
\BaseNTok{    if proportion not in (0,.) then percent = round(proportion * 100, .1);}
\BaseNTok{    if lowercl not in (0,.) then lowercl = round(lowercl * 100, .1); }
\BaseNTok{    else lowercl=0;}
\BaseNTok{    if uppercl not in (0,.) then uppercl = round(uppercl * 100, .1); }
\BaseNTok{    else uppercl=0;}
\NormalTok{run;}
\end{Highlighting}
\end{Shaded}

\href{https://documentation.sas.com/?cdcId=pgmsascdc\&cdcVersion=9.4_3.5\&docsetId=procstat\&docsetTarget=procstat_freq_syntax08.htm\&locale=en\#procstat.freq.freqbwald}{SAS doc}

  \bibliography{book.bib,packages.bib}

\end{document}
